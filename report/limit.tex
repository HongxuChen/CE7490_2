\section{Limitations}

The implementation of our RAID 6 storage bares the following limitations. 

First, since we have chosen the $\mathbf{GF}(2^8)$ as the field base, it can only support 255 drives at a time. In other words, the maximum number of data drives can be used to store the data is 255. In typical real world scenarios, 255 data drives are enough to store a chunk of data. Only in some extreme cases, one may need some more drives to store the data. In those cases, a large GF and a different generators (other than $\left\{02\right\}$) need to be chosen. However, to calculate the multiplication for 255 drive is very complexed without a lookup table. If the number of drive exceed 255, the calculation will cost even more resources. The performance of RAID 6 will become a serious issues.

Second, we have chosen to use padding for those files which do not have a good length to be partition into exact n number of data blocks. The approach works well for large files because the padding overhead is small compared with the file itself. However, for some very small files, adding the padding will become a high-cost operation, since the padding may be longer than the file itself. If this implementation is used to store huge number of small files, the total overhead will be much more higher, and thus the storage space is wasted. Some possible solutions may involve to pre-process the small files, which combine many files together to form a larger single file. Then using the padding technique will not result in a high cost.
