\section{Conclusion}

In this project, we have implemented the RAID 6 data storage to tolerant up to two drive faults at the same time. It allows the user to input a file with various length, and to specify the number of data drives they to to store the data across over. It then stores the file distributed over the virtual disk drives. Moreover, the implementation uses a pre-computed table for the Galois field multiplication to speed up the calculation, which overcome the bottleneck of RAID 6 architecture. Throughout this project, we have leant and mastered the knowledge of RAID storage, and used it to solve the practical problems.